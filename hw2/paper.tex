\documentclass[twocolumn]{IEEEtran}
\usepackage{epsfig}
%-----------------------

\begin{document}


\title{Insecurity of WEP}


\author{Arne Bjune and Vegar Engen \\ \texttt{\{arnedab,vegaen\}@cs.ucsb.edu}}

\markboth{Insecurity of WEP}{Bjune and Engen}

\maketitle

\begin{abstract}
802.11 has included Wired Equivalent Privacy (WEP) protocol, used to protect the 
link-layer communication from attacks. Several years ago critical 
security flaws were discovered, which compromises message transmission in WEP secured networks. 
In this paper we will discuss how WEP works, why it is broken, and how it the security issues
could have been avoided.
\end{abstract}

\section {Introduction}
\label{sec:introduction}

%% Something about increase of devices 

In contrast to 
a wired network, were it is a direct connection between the device and the router, wireless
networks have to send messages through the air, which makes it much easier to eavesdrop on the 
network traffic. This means a secured transmission is required. Through the original 802.11 
specs from 1999 \cite{IEEE:Fast}   \\

This paper is organized as following: We will start with the introduction in 
section \ref{sec:introduction}, then a brief explanation of what WEP is in 
section \ref{sec:whatiswep}. 


\section {What is WEP?}
\label{sec:whatiswep}

Wired Equivalent Privacy (WEP) is a protocol described in the documentation of 802.11 for 
wireless networks. WEP is used to protect link-layer transmissions from attacks. WEP uses a 
shared secret key \emph{k} to protect the data sent between the to parts. The protocol does also
make a checksum \emph{c(M)} of the message. The plain text that is going to be encrypted is 
\emph{P = M, c(M)}. 



\section {Technical Description}
\label{sec:technical_description}

WEP uses a stream cipher called RC4 to protect the data transmitted over a wireless link. RC4 was invented in 1987 by Ron Rivest and is widely used to protect network traffic. The seed for the RC4 stream cipher is 64 or 128 bits consisting of a 24 bit initialization vector and a 40/104 bit key. The key is normally represented as a 10/26 character hex values. A alternative way to represent the key is 5/13 ASCII characters but that reduces the keyspace with a factor of approximately 2.5 (95 vs 256 possible values for 8 bits).

A stream cipher produces a psudo random stream of bits which is then xored with the plaintext and transmitted to the receiver that runs the same algorithm and xores again to get the plaintext. A known problem with stream cipher is using the same key again. To avoid key stream reuse WEP was designed with a 24 bit initialization vector (IV) that is concatenated with the secret key to make the seed for the RC4 cipher. However the 802.11 standard says nothing about how IV should be chosen. Just using the same IV for every packet is a valid implementation according to the specification. A implementation choosing IV at random will have a collision every 5000 packets. 

The problem with two packets using the same IV is that they will use the same keystream. So its vulnerable to a Known Plaintext Attack.

C = P xor RC4(IV,key)
C1 xor C2 = (P1 xor RC4(IV,key) ) xor ( P2 xor RC4(IV,key)) = P1 xor P2

So if you know P1 it is trivial to find P2. 

\section {Conclusion}
\label{sec:conclusion}

Here we need space for the conclusion



\section {Future}
\label{sec:future}

WEP is now depreciated and has been replaced with WPA (802.11i draft) and WPA2 (802.11i-2004). WPA/WPA2 provides significantly improvements in security as well new ways of authenticating with the access point. The ability to have a central point  of authentication and support for RADIUS is crucial in a corporate environment.




\bibliographystyle{IEEEbib}
\bibliography{my}


\end{document}
