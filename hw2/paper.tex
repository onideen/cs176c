\documentclass[twocolumn]{IEEEtran}
\usepackage{epsfig}
%-----------------------

\begin{document}


\title{Insecurity of WEP}


\author{Arne Bjune and Vegar Engen \\ \texttt{\{arnedab,vegaen\}@cs.ucsb.edu}}

\markboth{Insecurity of WEP}{Bjune and Engen}

\maketitle

\begin{abstract}
802.11 has included Wired Equivalent Privacy (WEP) protocol, used to protect the 
link-layer communication from attacks. Several years ago critical 
security flaws were discovered, which compromises message transmission in WEP secured networks. 
In this paper we will discuss how WEP works, why it is broken, and how it the security issues
could have been avoided.
\end{abstract}

\section {Introduction}
\label{sec:introduction}

%% Something about increase of devices 

In the last decade the increase of mobile devices has been enormous, and to be connected to 
the Internet is required in a lot of daily activities.  with a wide specter of devices. To easily
be able to connect to Internet with this wide spectrum of devices, wireless networks have been
popular. But this new won freedom raises newer problems, is it secure to connect to a wireless
network? Since we in wireless networks are transmitting data through radio waves, in contrast to
a wired network were we transmitting data through a cable, the communication is a easier to 
intercept. Hence wireless network communication is in need for a way to protect 
the communication. \\


With the objective to enforce the security issues Wired Equivalent Privacy (WEP) protocol was 
introduced and described in 802.11 standard\cite{IEEE:Fast} for wireless networks. The most
important goal assigned to WEP was to ensure the users confidentiality from casual eavesdropping.
In addition two other main goals were added. First WEP was supposed to take care of access 
control, so only entrusted users were able to connect to the wireless infrastructure. The 
last goal was that WEP should ensure that data transmitted was not tampered with.   \\


Unfortunately, WEP was not able to live up to its expectation, since it came too short to
accomplish any of its main goals. Even though it was based on a well known and tested RC4 
stream cipher, a poor design choice was enough to make WEP contain major security flaws. Those
flaws make it possible to tampering, and eavesdrop on the wireless transmission. We will 
discuss attacks more in detail in section ??????. \\



This paper is organized as following: We will start with the introduction in 
section \ref{sec:introduction}, then a brief explanation of what WEP is in 
section \ref{sec:whatiswep}. 


\section {What is WEP?}
\label{sec:whatiswep}

Wired Equivalent Privacy (WEP) is a protocol described in the documentation of 802.11 for 
wireless networks. WEP is used to protect link-layer transmissions from attacks. WEP uses a 
shared secret key \emph{k} to protect the data sent between the to parts. The protocol does also
make a checksum \emph{c(M)} of the message. The plain text that is going to be encrypted is 
\emph{P = M, c(M)}. 



\section {Technical Description}
\label{sec:technical_description}

WEP uses a stream cipher called RC4 to protect the data transmitted over a wireless link. RC4 was invented in 1987 by Ron Rivest and is widely used to protect network traffic. The seed for the RC4 stream cipher is 64 or 128 bits consisting of a 24 bit initialization vector and a 40/104 bit key. The key is normally represented as a 10/26 character hex values. A alternative way to represent the key is 5/13 ASCII characters but that reduces the key space with a factor of approximately 2.5 (95 vs 256 possible values for 8 bits).





\section {Conclusion}
\label{sec:conclusion}

Here we need space for the conclusion



\section {Future}
\label{sec:future}

WEP is now depreciated and has been replaced with WPA (802.11i draft) and WPA2 (802.11i-2004). WPA/WPA2 provides significantly improvements in security as well new ways of authenticating with the access point. The ability to have a central point  of authentication and support for RADIUS is crucial in a corporate environment.




\bibliographystyle{IEEEbib}
\bibliography{my}


\end{document}
