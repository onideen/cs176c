\documentclass[twocolumn]{IEEEtran}
\usepackage{epsfig}
%-----------------------

\begin{document}


\title{Insecurity of WEP}


\author{Arne Bjune and Vegar Engen \\ \texttt{\{arnedab,vegaen\}@cs.ucsb.edu}}

\markboth{Insecurity of WEP}{Bjune and Engen}

\maketitle

\begin{abstract}
802.11 has included Wired Equivalent Privacy (WEP) protocol, used to protect the 
link-layer communication from attacks. Several years ago critical 
security flaws were discovered, which compromises message transmission in WEP secured networks. 
In this paper we will discuss how WEP works, why it is broken, and how it the security issues
could have been avoided.
\end{abstract}

\section {Introduction}
\label{sec:introduction}

%% Something about increase of devices 

In contrast to 
a wired network, were it is a direct connection between the device and the router, wireless
networks have to send messages through the air, which makes it much easier to eavesdrop on the 
network traffic. This means a secured transmission is required. Through the original 802.11 
specs from 1999 \cite{IEEE:Fast}   \\

This paper is organized as following: We will start with the introduction in 
section \ref{sec:introduction}, then a brief explanation of what WEP is in 
section \ref{sec:whatiswep}. 


\section {What is WEP?}
\label{sec:whatiswep}

Wired Equivalent Privacy (WEP) is a protocol described in the documentation of 802.11 for 
wireless networks. WEP is used to protect link-layer transmissions from attacks. WEP uses a 
shared secret key \emph{k} to protect the data sent between the to parts. The protocol does also
make a checksum \emph{c(M)} of the message. The plain text that is going to be encrypted is 
\emph{P = M, c(M)}. 



\section {Technical Description}
\label{sec:technical_description}

WEP uses a stream cipher called RC4 to protect the data transmitted over a wireless link. The seed for the stream cipher is 64 or 128 bits consisting of a 24bit initialization vector and a 40/104 bit key. The key is normally represented as a 10/26 character hex values.




\section {Conclusion}
\label{sec:conclusion}

Here we need space for the conclusion



\section {Future}
\label{sec:future}

Write about the future of the protocol




\bibliographystyle{IEEEbib}
\bibliography{my}


\end{document}
