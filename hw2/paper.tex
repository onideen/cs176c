\documentclass[twocolumn]{IEEEtran}
\usepackage{epsfig}
%-----------------------

\begin{document}


\title{Insecurity of WEP}


\author{Arne Bjune and Vegar Engen}

\markboth{Insecurity of WEP}
{Me and You}

\maketitle

\begin{abstract}
802.11 has included Wired Equivalent Privacy (WEP) protocol, used to protect the 
link-layer communication from attacks. Several years ago critical 
security flaws were discovered, which compromises message transmission in WEP secured networks. 
In this paper we will discuss how WEP works, why it is broken, and how it the security issues
could have been avoided.
\end{abstract}

\section {Introduction}
\label{sec:introduction}

An introduction is in place




\section {Technical Description}
\label{sec:technical_description}

WEP uses a stream cipher called RC4 to protect the data transmitted over a wireless link. RC4 was invented in 1987 by Ron Rivest and is widely used to protect network traffic. The seed for the RC4 stream cipher is 64 or 128 bits consisting of a 24 bit initialization vector and a 40/104 bit key. The key is normally represented as a 10/26 character hex values. A alternative way to represent the key is 5/13 ASCII characters but that reduces the keyspace with a factor of approximately 2.5 (95 vs 256 possible values for 8 bits).





\section {Conclusion}
\label{sec:conclusion}

Here we need space for the conclusion



\section {Future}
\label{sec:future}

WEP is now depreciated and has been replaced with WPA (802.11i draft) and WPA2 (802.11i-2004). WPA/WPA2 provides significantly improvements in security as well new ways of authenticating with the access point. The ability to have a central point  of authentication and support for RADIUS is crucial in a corporate environment.




\bibliographystyle{IEEEbib}
\bibliography{my}


\end{document}
