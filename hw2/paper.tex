\documentclass[twocolumn]{IEEEtran}
\usepackage{epsfig}
%-----------------------

\begin{document}


\title{Architectures for Unified Field Inversion with Applications 
in Elliptic Curve Cryptography}


\author{Author1 \& Author2
\thanks{Authors are with the
Department of Computer Science,
University of California, Santa Barbara, CA 93106.
E-mail: \texttt{\{you,me\}@cs.ucsb.edu}}
\thanks{This work is supported by Motorola.}
}

\markboth{IEEE Transactions On Computers, Vol. 51, No. 5, May 2002}
{Me and You}

\maketitle

\begin{abstract}
We are presenting two new inversion algorithms for binary extension
and prime fields which are slightly modified versions of the
Montgomery inverse algorithm. An hardware architecture implementing
these algorithms is also introduced, in which the field elements
are represented using a multi-word format. This feature allows a
scalable and unified architecture which operates in a broad range 
of precision, which has advantages in elliptic curve cryptography.
\end{abstract}

\section {Introduction}

The basic arithmetic operations (i.e. addition, multiplication, and 
inversion) in prime and binary extension fields, $GF(p)$ and $GF(2^n)$, 
have several applications in cryptography, such as decipherment 
operation of RSA algorithm \cite{QC82:Fast}, Diffie-Hellman key exchange 
algorithm \cite{DH76:New}, the Government Digital Signature Standard 
\cite{NIST91:DSS} and also elliptic curve cryptography 
\cite{K87:Elliptic,M93:Elliptic}. Recently, speeding up inversion operation
in both fields has been gaining some attention since inversion is the
most time consuming operation in elliptic curve cryptographic algorithms 
when affine coordinates are selected \cite{K95:The,SOOS95:Fast,%
K99:Fast,SK00:The,H01:Efficient}.

In this paper, we will give and analyze multiplicative inversion algorithms
for $GF(p)$ and $GF(2^n)$ which allow very fast and area-efficient, unified
and scalable hardware
implementations. The algorithms are based on the Montgomery inverse algorithms 
given in \cite{K95:The}. 

\section {The Montgomery Inversion Algorithm}

The following algorithm performs the Montgomery inversion in $GF(2^n)$.
However, the Phase II of the algorithm is omitted since it is not
relevant to this paper, and a similar algorithm is in \cite{SK00:The}.

\noindent \textbf{Algorithm A} \\
\textbf{Input:} $a(x)$ and $p(x)$, where $deg(a(x)) < deg(p(x))$ \\
\textbf{Output:} $r(x)$ and $k$, where $r=a(x)^{-1} x^k \pmod{p(x)}$  
and $deg(a(x)) \leq k \leq deg(p(x))+deg(a(x))+1$

\begin{tabbing}
\hspace*{0cm} \= \hspace{0.5cm} \= \hspace{1ex} \= \hspace{1ex} \= \kill
\> 1:  \> $u(x) := p(x)$, $v(x) := a(x)$, $r(x) := 0$, and $s(x) := 1$\\
\> 2:  \> $k:=0$\\
\> 3:  \> while ($v(x) != 0$) \\
\> 4:  \> \> if $u(0) = 0$ then $u(x) := u(x)/x$, $s(x) := xs(x)$ \\
\> 5:  \> \> else if $v(0)=0$ then $v(x) := v(x)/x$, $r(x) := xr(x)$ \\
\> 6:  \> \> else if $deg(u(x)) > deg(v(x))$ then \\
\>	\>  \> \> $u(x):= (u(x)+v(x))/x$ \\
\> \>  \> \> $r(x) := r(x) + s(x)$ \\
\> \>  \> \> $s(x) := xs(x)$ \\
\> 7:  \> \> else $v(x) := (v(x)+u(x))/x$ \\
\> \>  \> \> $s(x) := s(x)+r(x)$ \\
\> \>  \> \> $r(x) := xr(x)$ \\
\> 8:  \> \> $k:=k+1$\\
\> 9:  \> if $deg(r(x)) = deg(p(x))$ then $r(x) := r(x)+p(x)$ \\
\> 10: \> return $r(x)$ and $k$
\end{tabbing}
The following properties are observed: 

\noindent $\bullet$ 
If $deg(p(x))>deg(a(x))>0$, then the degrees of intermediate binary polynomials
$r(x)$, $s(x)$, $u(x)$, and $v(x)$ in the
Montgomery inverse algorithm are always in the range $[0, deg(p(x))]$.

\noindent $\bullet$ 
If $p(x)$ is an irreducible polynomial, and $deg(p(x))>deg(a(x))>0$, then
$n + 1 < k \leq deg(a(x)) + n + 1$.  

\noindent $\bullet$ 
If $p(x)$ is an irreducible polynomial, and $deg(p(x))>deg(a(x))>0$, then 
Phase I of Montgomery inverse algorithm for $GF(2^n)$ returns $a(x)^{-1}x^k$
$\pmod{p(x)}$.

Additions and subtractions in the original algorithm are replaced with 
additions without carry in $GF(2^n)$ version of the algorithm. Since it is
possible to perform addition (and subtraction) with carry and addition
without carry in a single arithmetic unit, this difference does not 
cause a change in the control unit of a possible unified hardware implementation.  
On the other hand, the algorithm for $GF(2^n)$ differentiates from the original 
algorithm in Step~6, in which the degrees of $u(x)$ and $v(x)$ are compared.
In order to have a unified architecture,
we propose a slight modification in the original algorithm
for $GF(p)$ which is given in the following section.  

\section {A Variation of Montgomery Inversion Algorithm}

We propose to modify Step 6 of the algorithm given in \cite{K95:The} in a
such way that instead of comparing $u$ and
$v$, number of bits needed  to represent them are compared.
The proposed modifications can be seen in Step~6 and
Step~7.a of the modified algorithm given below: 

\medskip 

\noindent \textbf{Algorithm B} \\
\textbf{Input:} $a \in [1,p-1]$ and $p$ \\
\textbf{Output:} $r \in [1,p-1]$ and $k$, where
$r=a^{-1}2^k \pmod{p}$ and $n \leq k \leq 2n$

\begin{tabbing}
\hspace*{0cm} \= \hspace{0.8cm} \= \hspace{0.5cm} \= \hspace{0.5cm} \= \kill
\> 1:  \> $u:=p$, $v:=a$, $r:=0$, and $s:=1$\\
\> 2:  \> $k:=0$\\
\> 3:  \> while ($v > 0$) \\
\> 4:  \> \> if $u$ is even then $u:=u/2$, $s:=2s$ \\
\> 5:  \> \> else if $v$ is even then $v:=v/2$, $r:=2r$ \\
\> 6:  \> \> else if $bitsize(u) > bitsize(v)$ then \\
\>     \> \> \> $u:=(u-v)/2$, $r:=r+s$, $s:=2s$ \\
\> 7:  \> \> else then \\
\>     \> \> \> $v:=(v-u)/2$, $s:=s+r$, $r:=2r$ \\
\> 7.a:\> \> \> if $v<0$ then $v := -v$, $s := -s$ \\ 
\> 8:  \> \> $k:=k+1$\\
\> 9:  \> if $r<0$ then \\
\> 9.a:\> \> if $r \leq -p$ then $r:=r+p$ \\
\> 9.b:\> \> return $r:=-r$ \\
\> 10: \> else \\ 
\> 10.a: \> \> if $r \geq p$ then $r:=r-p$ \\
\> 10.b:\> \> return $r:=p-r$ and $k$
\end{tabbing}

When corrections in Step~7.a are executed, the effect is multiplying both sides of
the invariant by $-1$. Therefore, new invariant when $s<0$ is given as
$-p=us+vr$. While $u$ and $v$ remain to be positive integers, $s$ and $r$ 
might be positive or negative. Therefore, we need to alter the final reduction
steps to bring $r$ in the correct range, which is $[0, p)$. The range of $s$ and $r$
are $[-p, p]$ and $[-2p, 2p]$, respectively. As a result we need to use one more bit
to represent $s$ and $r$ than in the original algorithm. The advantage of this
version of the algorithm will be discussed in the next section.

\section {Hardware Architecture}

Scalability of the arithmetic modules is important in cryptographic 
context since it allows to increase the key length when the need
for more security arises without having to modify or re-design
the cryptographic unit. The scalability of the inverter unit can easily 
be achieved by using shifter and adder units which handle only certain
number of bits of the operands at a time. One addition (or shift) operation,
therefore, in the corresponding field takes more than one clock cycle.
The number of bits that the unit operate on is referred as {\it word} and its
length can be determined or adjusted with respect to 
given area, speed or latency requirements. 

The algorithms~B and C can be implemented in a unified hardware architecture 
provided that a dual-field adder/subtractor (DFA/S) that operates in both fields
is available. In order for the inverter unit to be scalable,
The DFA/S is designed to handle words of finite number of bits at a clock
cycle, therefore we call them word DFA/s(WDFA/S).

Except the final correctional steps (steps~9 through 11), 
the main loops of the Algorithm~A and Algorithm~B 
can be implemented in the same
hardware unit. The only difference in the main loops of the 
two algorithms is that the Algorithm~B has the extra Step~7.a. 
However, this extra step neither necessitates a major change 
in the circuitry nor introduces any extra clock cycle in the 
computation. Algorithm~B replaces integer comparison operation
of the original algorithm with just one bitsize comparison.
In exchange for that, some of the intermediate variables
take negative integer values. For example, the variables
$v$ and $s$ may have to change sign in Step~7.a if the subtraction 
operation in Step~7 produce a negative result. Taking two's
complement of these two variables may re-introduce the clock
cycles we saved by eliminating integer comparison operation
in Step~6 of the original algorithm \cite{K95:The}. On the other hand, When 
variable $v$ turns out to be a negative number as a result
of the subtraction in Step~7, we may keep it as negative in two's
complement representation. In the
next iteration in the loop, it can easily be seen that Step~5
or Step~6 is executed. 
Sign change of the variable may be performed at the
same time as the subtraction operation in the subsequent Step~6.

On the other hand, the magnitutes of $r$ and $s$ cannot easily be 
determined. Therefore, we need to devise a method in order to avoid
taking two's complement of $s$ in Step~7.a. We propose
to maintain one extra bit for each of the variables $s$ and $r$ which 
holds extra sign information for them. We call this extra sign bit 
as {\it correct sign} ($CS$) of the variable. These variables can
be kept as negative (in two's complement representation) or positive, 
however, their real sign
is determined by the value in correct sign bit. If their actual 
sign is different from the one in the correct sign bit,
the sign must be flipped. On the other hand, taking two's complement
when this happens is not desirable since we want to avoid
the extra clock cycles it introduces. The actual and correct
signs of a variable determine the way we execute the
addition operation $x:=r+s$ in Steps~6 and~7. Assuming that
$S_x$ and $CS_x$ are the actual and correct sign of the variable
$x$ respectively, this operation is performed as in the following:

\medskip

\noindent \textbf{Algorithm C} \\
\textbf{Input:} $r$, $s$, $S_r$, $S_s$, $CS_r$, and $CS_s$ \\
\textbf{Output:} $x := r+s$, $S_x$, and $CS_s$

\begin{tabbing}
\hspace*{0cm} \= \hspace{0.5cm} \= \hspace{1ex} \hspace{1ex} \= \kill
\> 1:   \> if $S_r=CS_r$ and $S_s=CS_s$ then \\
\> 1.a: \> \> $x:=s+r$ and $CS_x:=S_x$ \\ 
\> 2:   \> else if $S_r=CS_r$ and $S_s=\bar{CS_s}$ then \\
\> 2.a: \> \> $x:=r-s$ and $CS_x:=S_x$ \\ 
\> 3:   \> else if $S_r=\bar{CS_r}$ and $S_s=CS_s$ then \\
\> 3.a: \> \> $x:=s-r$ and $CS_x:=S_x$ \\ 
\> 4:   \> else $S_r=\bar{CS_r}$ and $S_s=\bar{CS_s}$ then\\
\> 4.a: \> \> $x:=s+r$ and $CS_x:=\bar{S_x}$ 
\end{tabbing}

\section {Complexity Analysis of the Unified Inverter}

Assuming that we have two WDFA/S in our design, the total computation time 
of inversion in terms of total clock cycle count can be computed using the 
formula $T = k \cdot (e+1)$,
where $k$ is the iteration index in the main loop of the algorithms, 
$e=\lceil \frac {n+1} {w} \rceil$ is the number of words and $w$ is 
the word length.

Based on these experimental values we calculated the estimated 
execution time in terms of number of clock cycles for inversion operation 
using word length 32. We summarized the results in Table~1.
Table~1 also includes the clock cycle count estimates for
the modular multiplication operation for the same precisions, which is 
assumed to be performed using unified and scalar Montgomery modular 
multiplication unit proposed in \cite{STK00:A} with 7 pipeline 
stages and 32-bit word size. The ratio of inversion time
to multiplication time, which is important in the decision 
whether affine or projective coordinates are to be
employed in elliptic curve cryptography, is also included in the
table. It is argued in \cite{LD99:Fast} that for binary extention
fields $GF(2^k)$ projective 
coordinates, which does not entail fast execution of inversion
operation, perform better than the affine coordinates when
inversion operation is more than 7 times slower than the multiplication
operation. Similarly, our calculations show that this ratio is about 9 
for prime field $GF(p)$. As can be observed in Table~1 the ratio stays lower
than 7 for the precisions of interest to the elliptic curve 
cryptography.

\begin{center}
\textbf{Table 1:} Estimated clock cycles for inversion and the ratio 
to the multiplication operation. \\[1em]
\begin{tabular}{|c|c|c|c|c|c|c|c|c|c|c|c|c|c|} \hline
bitsize & Inversion & Multiplication & Ratio  \\ \hline
160 & 1368  & 327 & 4.18 \\ \hline
192 & 1911  & 398 & 5.00 \\ \hline
224 & 2544  & 469 & 5.42 \\ \hline
256 & 3276  & 526 & 6.23 \\ \hline
\end{tabular}
\end{center}

In Figure~1 and Figure~2, hardware realizations of the operations $(u-v)/2$ 
and $r+s$ are shown, respectively. In Figure~1, the building block~A simply 
seperates the least significant bit from the rest of the result bits, which are 
to be kept in the latch one clock cycle in order to perform shift operation.
In the next clock these bits are combined with the least significant bit of 
the current result, which is placed in the most significant position of the 
final resulting word, in block~B. The block~C of the Figure~1, is used to 
connect the register outputs to the correct input of the adder/subtractor unit.
The circuit in Figure~2 performs 
two operations: $r+s$ and $2r$(or $2s$). The register content, which is to be
shifted left by one bit, is available at the output of block~D. The block~D
is also used to connect the register outputs to the correct input of the 
adder/subtractor unit. Blocks~A and B are used to shift a word in each 
clock cycle. Block~C directs the results of the two operation 
($r+s$ and $2r$(or $2s$)) to the appropriate registers. 

\bigskip
\bigskip

\centerline{\textbf{Figure 1:} Hardware realization of $(u-v)/2$.}
\medskip
\epsfxsize=3.1in
\centerline{\epsffile{f1.eps}}

\bigskip
\bigskip

\centerline{\textbf{Figure 2:} Hardware realization of $r+s$.}
\medskip
\epsfxsize=3.1in
\centerline{\epsffile{f2.eps}}

\bibliographystyle{IEEEbib}
\bibliography{my}


\end{document}

